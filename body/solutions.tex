\begin{exo}
Soient $\varepsilon, L>0$ et $\fl$ l'ensemble des applications $L$-Lipschitziennes de $\intff{0}{1}$
 dans $\R$. Soit $\A$ un algorithme renvoyant, pour toute fonction $ f \in \fl$,
  un nombre $\xopt \in \intff{0}{1}$ tel que $\displaystyle f \left( \xopt \right) \leq \min _ { [ 0,1 ] } f + \varepsilon$. On suppose que $\A$ fait uniquement appel à l’oracle d’ordre zéro.
\end{exo}

\begin{qst}
Montrer que $\A$ doit faire appel à l’oracle d’ordre zéro au moins $\Omega ( L / \varepsilon )$ fois. \textit{Indice: Considérer la fonction nulle, et construire une fonction $g \in \fl$ valant zéro en tous les points en lesquels l’oracle d’ordre zéro est appelé pour la fonction nulle, dont le minimum est le plus petit possible.}
\end{qst}

\begin{rep}

Tout d'abord, notons que pour $ f \in \fl$ le problème d'optimisation $ \displaystyle \min _ { [ 0,1 ] } f$ est bien définie car on optimise une fonction continue\footnote{rappelons que lipschitzienne entraîne uniformément continue qui entraîne continue.} sur un compat.

Appliquons comme indiqué l'algorithme $\A$ à la fonction nulle sur $\intff{0}{1}$ notée $f$. Soient $x_1 \leq x_2 \ldots \leq x_n$ la suite de points de $\intff{0}{1}$ utilisés lors des appels à l'oracle d'ordre zéro. Posons $x_0 = 0$ et $x_{n+1}=1$ et introduisons:
  $$\alpha := \max_{0\leq i \leq n} \abs{x_{i+1}-x_{i}}
  , \quad  k := \argmax_{0\leq i \leq n} \abs{x_{i+1}-x_{i}}
  \quad \text{ et } \quad \xtilde := \frac{x_{k+1}+x_{k}}{2}
  $$

  Soit $g:\intff{0}{1} \mapsto \R$, la fonction continue et affine par morceaux de pente
  $-L$ et $L$ sur $\intff{x_k}{\xtilde}$ et $\intff{\xtilde}{x_{k+1}}$ respectivement. $g$
  s'écrit simplement:
  $$g(x)= \min \acc{ 0,L\prt{\abs{x-\xtilde} - \abs{x_k-\xtilde}} }$$

Notons que l'on a bien $g\in \fl$.

  Comme l'algorithme $\A$ dépend uniquement des valeurs $f(x_i)=g(x_i)$ pour $i=1$ à $n$,
  la sortie de $\A$ noté $\xopt$\footnote{$\xopt=x_{j}$ pour un certain $j\in \acc{1,\ldots,n}$.}  sera la même pour $f$ ou $g$. Par définition de $\A$, on a que :
   $ \displaystyle g \left( \xopt \right) \leq \min _ { [ 0,1 ] } g + \varepsilon$,
 ce qui s'écrit ici
 \begin{equation}\label{eq:g}
   0 \leq -L \frac{\alpha}{2} + \varepsilon.
 \end{equation}
   De plus on a: $ 1= \sum_{i=0}^n x_{i+1}-x_i \leq (n+1) \cdot \alpha$,
   d'où
\begin{equation}\label{eq:alpha}
\alpha \geq \frac{1}{n+1}.
\end{equation}
   En combinant\eqref{eq:g} et \eqref{eq:alpha}, on en déduit
   \begin{equation}
     n \geq \frac{L}{2 \varepsilon}-1,
   \end{equation}
   d'où le résultat.
\end{rep}


\begin{qst}
En déduire la complexité optimale sur $\fl$, en fonction de $L$ et $\varepsilon$, pour les algorithmes ne faisant appel qu’à l’oracle d’ordre zéro.
\end{qst}

\begin{rep}

Notons $x_k := \frac{k}{n}$ pour $k=0$ à $n$ et introduisons l'algorithme $\A$
 renvoyant $\displaystyle \xopt = \argmin_{x \in \acc{x_i}_{i=0}^n} f(x)$.
 Soit $\xtilde$ un minimum de $f$ sur $\intff{0}{1}$. Notons qu'il existe $i$ tel
  que $ x_i \leq \xtilde \leq x_{i+1}$.

  En utilisant le fait que f est $L$-lipschitzienne, on a :
  \begin{align*}
    \abs{f(\xtilde)-f(x_i)} &\leq \abs{f(x_{i+1})-f(x_i)}\\
    &\leq L \abs{x_{i+1}-x_i}\\
    &= \frac{1}{n} \cdot L
  \end{align*}
  En prenant $n\geq \frac{1}{\varepsilon}\cdot L$ suffisamment grand, on a bien
la précision attendue pour $\A$, ce qui permet de conclure à l'optimalité de $\bigo{L/\varepsilon}$. 
\end{rep}
